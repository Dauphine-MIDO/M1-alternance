%\documentclass[minimal,compress, draft]{beamer} %draft pour pas mettre image et compil rapidement
\documentclass[minimal,compress]{beamer}
%\documentclass[minimal,compress]{thesis}


%virer la barre moche
\setbeamertemplate{navigation symbols}{}


%gets rid of bottom navigation bars
%\setbeamertemplate{footline}[page number]{}

\usepackage[utf8]{inputenc}
%\usepackage[english]{babel}
\usepackage[french]{babel}
\usepackage{hyperref}
\usepackage{graphicx}
\usepackage{eso-pic}	%put logo anywhere
\usepackage{xcolor}


\definecolor{mydarkred}{rgb}{0.40,0,0}
\definecolor{myred}{rgb}{0.50,0.17,0.17}
\definecolor{mydarkgray}{rgb}{0.33,0.33,0.33}
\definecolor{mygray}{rgb}{0.47,0.47,0.47}
\definecolor{mydarkblue}{rgb}{0.16,0.26,0.46}
\definecolor{myblue}{rgb}{0.43,0.50,0.64}
\definecolor{mydarkgreen}{rgb}{0.16,0.46,0.26}
\definecolor{mygreen}{rgb}{0.43,0.64,0.50}

\beamerboxesdeclarecolorscheme{redblock}{mydarkred}{myred!50!white}
\beamerboxesdeclarecolorscheme{greenblock}{mydarkgreen}{mygreen!50!white}

%http://mcclinews.free.fr/latex/beamergalerie/completsgalerie.html
%\usetheme{Dresden}
%\usetheme{Frankfurt}
%\usetheme{Montpellier}
\usetheme{monTheme}			%http://titilog.free.fr/index.htm


%entete pied page http://mcclinews.free.fr/latex/beamergalerie/outerdetails.html
%\useoutertheme{shadow}
%\useoutertheme[subsection=false]{Smoothbars}
%\useoutertheme{infolines}
%\useoutertheme{miniframes}
%\useoutertheme{smoothtree}
%\useoutertheme{tree}

%ensemble des sections dans la headline
%http://mcclinews.free.fr/latex/beamermodif/beamersection.html
\defbeamertemplate*{headline}{}
{%
  \begin{beamercolorbox}{section in head/foot}
    \insertsectionnavigationhorizontal{\paperwidth}{}{}
  \end{beamercolorbox}%
 }

%\defbeamertemplate*{headline}{miniframes theme}
%{%
%  \begin{beamercolorbox}{section in head/foot}
%    \insertsectionnavigationhorizontal{\paperwidth}{}{}
%  \end{beamercolorbox}%
%}

%shadow sous la healine depuis le beamerthemeshadow
%\setbeamercolor{frametitle}{parent=subsection in head/foot}
%\setbeamercolor{frametitle right}{parent=section in head/foot}
%\pgfdeclarehorizontalshading[frametitle.bg,frametitle right.bg]{beamer@frametitleshade}{\paperheight}{%
%  color(0pt)=(frametitle.bg);
%  color(\paperwidth)=(frametitle right.bg)}
%\AtBeginDocument{
%  \pgfdeclareverticalshading{beamer@topshade}{\paperwidth}{%
%    color(0pt)=(bg);
%    color(4pt)=(black!50!bg)}
%}
%\addtobeamertemplate{headline}
%{}
%{%
%  \vskip-0.2pt
%  \pgfuseshading{beamer@topshade}
%  \vskip-2pt
%}






\setbeamertemplate{frametitle}[default]
\setbeamertemplate{blocks}[rounded][shadow=true]

\usefonttheme{structurebold}
\setbeamerfont{section in toc}{series=\bfseries}
\setbeamerfont{alerted text}{series=\bfseries}
\setbeamerfont{quote}{series=\normalfont}

%\setbeamercolor{section in head}{fg=lightgray,bg=mydarkgray}
\setbeamercolor{section in head/foot} {fg=lightgray,bg=mydarkgray}
%\setbeamercolor{section in foot} %{fg=lightgray,bg=mydarkgray}
%{fg=white,bg=white}
\setbeamercolor{subsection in head/foot}{fg=mydarkred,bg=}
\setbeamercolor{titlelike}{fg=mydarkred,bg=}
\setbeamercolor{section in toc}{fg=mydarkred,bg=}
\setbeamercolor{subsection in toc}{fg=black!,bg=}
\setbeamercolor{item}{fg=mydarkred,bg=}
\setbeamercolor{subitem}{fg=mydarkgray,bg=}
\setbeamercolor{subsubitem}{fg=myred,bg=}
\setbeamercolor{navigation symbols}{fg=mygray,bg=}
\setbeamercolor{block title example}{fg=mydarkred,bg=}
\setbeamercolor{block title}{fg=mydarkred,bg=}
\setbeamercolor{block body}{fg=,bg=}
%\setbeamercolor{frametitle}{fg=mydarkgray,bg=}
\setbeamercolor{frametitle}{fg=mydarkred,bg=}
\setbeamercolor{alerted text}{fg=mydarkred,bg=}

%\pgfpagesuselayout{4 on 1}[a4paper,border shrink=5mm,landscape]

%In order to get a full page printable presentation (i.e. auto resized to A4 paper)
%\pgfpagesuselayout{resize to}[a4paper,border shrink=0mm,landscape]



%logo dauphine
\newcommand\AtPagemyUpperLeft[1]{\AtPageLowerLeft{%
\put(\LenToUnit{0.75\paperwidth},\LenToUnit{0.89\paperheight}){#1}}}
\AddToShipoutPictureFG{
  \AtPagemyUpperLeft{{\includegraphics[width=3cm,keepaspectratio]{dauphine_psl2018}}}
}%


%numero pages
\addtobeamertemplate{footline}{\hspace{1em}\insertshorttitle\hfill\insertframenumber/\inserttotalframenumber\hspace{1em}\null}


%footer
%\makeatother
%\setbeamertemplate{footline}
%{
%  \leavevmode%
%  \hbox{%
%%  \begin{beamercolorbox}[wd=.4\paperwidth,ht=2.25ex,dp=1ex,center]{author in head/foot}%
%%    \usebeamerfont{author in head/foot}\insertshortauthor
%%  \end{beamercolorbox}%
%  \begin{beamercolorbox}[wd=.6\paperwidth,ht=2.25ex,dp=1ex,center]{title in head/foot}%
%    \usebeamerfont{title in head/foot}\insertshorttitle\hspace*{3em}
%    \insertframenumber{} / \inserttotalframenumber\hspace*{1ex}
%  \end{beamercolorbox}}%
%  \vskip0pt%
%}
%\makeatletter

%%%%%%%%%%%%%%%%%%%%%%%%%%%%%%%%%%%%%%%%%%%%%%%%%%%%%%%%%%%%%%%%%%%%%%%%%%%%%% 

\title[M1 MIAGE Apprentissage]
{MIAGE (Méthodes Informatiques Appliquées à la Gestion des Entreprises)}

\author{M1 - Formation en apprentissage}
\institute{}%
\date{}



%%%%%%%%%%%%%%%%%%%%%%%%%%%%%%%%%%%%%%%%%%%%%%%%%%%%%%%%%%%%%%%%%%%%%%%%%%%%%% 

\begin{document}



\frame[plain]{
 \titlepage
 
 
 \begin{tabular}{ll}
Responsable pédagogique Dauphine: & \textbf{Olivier Cailloux} \\
Chargée de mission CFA AFIA: & \textbf{Patricia Lavagna} \\
Administration: & \textbf{Siham Teguia}
 \end{tabular}
}




\begin{frame}{Les disciplines}
\begin{itemize}
\item Technologies de l’\alert{informatique} et de la \alert{gestion}

\item Modélisation de \alert{systèmes d’information} et méthodes de \alert{conduite de projet}

\item Fonctionnement des organisations, leurs structures, et leurs impératifs stratégiques

\item L’anglais %{\scriptsize (plus une deuxième langue vivante)}

\item \alert{Professionnalisation} avec la formation par apprentissage
\end{itemize}
\end{frame}

\begin{frame}{Entreprises d'accueil}

\begin{itemize}
\item Banques, Assurances, Sociétés de Services, Startups, Logistique,...
\item Par exemple:
\end{itemize}

\centering
\begin{minipage}{.35\textwidth}
 Caisse des dépôts
 
 Dassault
 
 Air France
 
BNP Paribas

 EDF
 

\end{minipage}
\begin{minipage}{.35\textwidth}
 LCL
 
 Société Générale
 
GFI

 Natixis
 
 Canal+

\end{minipage}



\end{frame}

\begin{frame}{Organisation}
\begin{itemize}
\item Apprentissage
\begin{itemize}
\item 1 semaine entreprise / 1 semaine université
\item 	Maitre d’apprentissage en entreprise + tuteur enseignant
\item 	Examen au fil de l’eau
\item 	Présence obligatoire cours/TD/TP

\end{itemize}
\item Promotion
\begin{itemize}
\item 28 apprentis
\item L3 apprentissage  en grande partie. Quelques places pour le L3 formation initiale et les externes à l’université Paris Dauphine
\end{itemize}
\end{itemize}
\end{frame}



\begin{frame}{Programme S1}
\begin{itemize}
\item \alert{Bloc Fondamental 1} Semestre 1 (22 ECTS)
\begin{itemize}
\item Programmation Objet Avancée (4 ECTS)
\item Intelligence Artificielle (3 ECTS)
\item Introduction au Machine Learning (4 ECTS)
\item Systèmes de Gestion de Bases de Données  (4 ECTS)
\item Systèmes d’Information Avancés 1 (3 ECTS)
\item Systèmes et algorithmiques répartis (4 ECTS)
\end{itemize}
\end{itemize}
\end{frame}

\begin{frame}{Programme S1}
\begin{itemize}
\item \alert{Bloc communication} Semestre 1 (8 ECTS)
\begin{itemize}
\item Notions générales de droit (3 ECTS)
\item Marketing (3 ECTS)	
\item Anglais 1 (2 ECTS)
%\item Deuxième langue vivante (3 ECTS)
\end{itemize}
\end{itemize}
\end{frame}


\begin{frame}{Programme S2}
\begin{itemize}
\item \alert{Bloc Fondamental 2} Semestre 2 (16 ECTS)
\begin{itemize}
\item Analyse Financière (3 ECTS) 
\item Systèmes d’Information Avancés 2 (3 ECTS)
\item Éthique en informatique et protection des données (1 ECTS)
\item Jeux d’entreprise (1 ECTS)
\item Organisation et communication (3 ECTS)
\item Programmation Web (3 ECTS)
\item Anglais 2 (2 ECTS)
\end{itemize}
\end{itemize}
\end{frame}

\begin{frame}{Programme S2}
\begin{itemize}
\item \alert{Options} Semestre 2 (6 ECTS) % (une \textcolor{mydarkgreen}{verte} et une \textcolor{mydarkblue}{bleue})
\begin{itemize}
\item {Marchés financiers (3 ECTS)} 
\item Décision collective, décision multicritère (3 ECTS)
\item {Ordonnancement et  gestion de production (3 ECTS)}
\item Sécurité et réseaux (3 ECTS)
\end{itemize}
\item 2 options à choisir.
\end{itemize}
\end{frame}


\begin{frame}{Programme}

\begin{itemize}
\item \alert{Bloc entreprise} (8 ECTS)
\end{itemize}

\begin{itemize}
\item Modalités:
\begin{itemize}
\item Mémoire sur une problématique rencontrée dans le cadre de la mission
\item Soutenance (première semaine de septembre)
\end{itemize}
\item Évaluation:
\begin{itemize}
\item Pertinence du sujet
\item Qualité du travail réalisé (recherche, analyse et synthèse)
\item Qualité de la rédaction du mémoire
\item Travail en entreprise
\end{itemize}

\end{itemize}

\end{frame}


\begin{frame}{Contrôle des connaissances}
\begin{itemize}
\item Validation de l’année:
\begin{itemize}
\item Note finale de \alert{chacun des blocs} $\geq$ \alert{10}/20
\item Note finale de \alert{chaque UE} $\geq$ \alert{5}/20
\item Valider 60 ECTS.
\end{itemize}
\item Seconde session:
\begin{itemize}
\item Repasser les UEs avec une note $<$ \alert{5/20}
\item Compensation des UE dans un bloc
\item Choix de repasser ou non une UE avec note finale dans [5,10[
\end{itemize}
\end{itemize}
\end{frame}

\begin{frame}{M2 MIAGE en apprentissage}

\begin{itemize}
\item Accès à la \alert{2ème année} du Master Informatique des Organisations.
\item Spécialités:
\begin{itemize}
\item \alert{MIAGE-IF} (Informatique pour la finance):
\begin{itemize}
\item Forme des informaticiens dans les métiers de la \alert{banque et de l’assurance}
\end{itemize}
\item \alert{MIAGE-SITN} (Sys. d’info. et tecno. nouv.)
\begin{itemize}
\item Forme des professionnels en conception et mise en œuvre de \alert{systèmes d’information}
\end{itemize}
\item \alert{ID-MIAGE} (Informatique Décisionnelle)
\begin{itemize}
\item Forme des informaticiens en \alert{aide à la décision et en Informatique Décisionnelle}
\end{itemize}
\end{itemize}
\end{itemize}

\end{frame}





%%%%%%%%%%%%%%%%%%%%%%%%%%%%%%%%%%%%%%%%%%%%%%%%%%%%%%%%%%%%%%%%%%%%%%%%%%%%%% 

\end{document}
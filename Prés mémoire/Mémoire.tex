\RequirePackage[l2tabu, orthodox]{nag}
\documentclass[french]{beamer}
\input{preamble/packages}
\input{preamble/redac}
\input{preamble/math_basics}
\input{preamble/math_mine}

%\setbeamertemplate{headline}[singleline]
%\setbeamertemplate{footline}[authortitle]

\title{Le mémoire en M1 par alternance}
\subject{Enseignement}
\keywords{consignes, rédaction}
\author{Olivier Cailloux}
\institute[LAMSADE]{\inst{1} LAMSADE, Université Paris-Dauphine}
\date{Version du \today}

\begin{document}
\begin{frame}[plain]
	\tikz[remember picture,overlay]{
		\path (current page.south west) node[anchor=south west, inner sep=0] {
			\includegraphics[height=8mm]{Dauphine-Noir.png}
		};
		\path (current page.south east) node[anchor=south east, inner sep=0] {
			\includegraphics[height=1cm]{LAMSADE95.jpg}
		};
		\path (current page.south) ++ (0, 4em) node[anchor=south, inner sep=0] {
			\scriptsize\textcolor{blue}{\url{https://github.com/Dauphine-MIDO/M1-alternance}}
		};
	}
	\titlepage
\end{frame}
\addtocounter{framenumber}{-1}

\section{Sujet}
\begin{frame}
	\frametitle{Sujet}
	Le sujet doit être une problématique
	\begin{block}{Problématique, kesako ?}
		Un problème
		\begin{itemize}
			\item abstrait
			\item bien posé
			\item admettant une multiplicité de solutions
		\end{itemize}
	\end{block}
	Voyons cela en détail…
\end{frame}

\begin{frame}
	\frametitle{Problème abstrait}
	\begin{itemize}
		\item Plus général que la mission en entreprise
		\item Couvrir diverses instanciations concrètes possibles
		\item Dans d’autres situations, d’autres entreprises
	\end{itemize}
	\begin{example}[Problème abstrait]
		Implémentation d’une technique de gestion de projet agile dans une PME
	\end{example}
	\begin{example}[Problème trop concret]
		Implémentation des conseils fournis dans le chapitre 4 du livre \emph{Scrum pour les nuls} dans la boucherie Sanzo du 104 rue de Paris en 2023
	\end{example}
\end{frame}

\begin{frame}
	\frametitle{Problème bien posé}
	\begin{itemize}
		\item Permet analyse de portée limitée
		\item Analyse de façons de répondre au problème
		\item Permettant d’analyser dans quelle mesure on a répondu au problème !
	\end{itemize}
	\begin{example}[Problème trop vague]
		Dans notre entreprise on ne gère pas bien nos projets
	\end{example}
	\begin{example}[Problème mieux posé]
		Comment implémenter une technique de gestion de projet agile dans une situation de… pour…
	\end{example}
\end{frame}

\begin{frame}
	\frametitle{Multiplicité de solutions}
	\begin{itemize}
		\item Multiples possibilités de réponse a priori
		\item Pas évident au départ quelle est la meilleure
		\item Analyse de façons de répondre au problème
		\item Permettant d’analyser dans quelle mesure on a répondu au problème !
	\end{itemize}
	\begin{example}[Solution évidente]
		Bien considérer ses employés permet-il de réduire l’absentéisme et augmenter le bien-être au travail ?
	\end{example}
	\begin{example}[Solutions multiples]
		Quelle gestion de l’avancement de carrière permet de réduire l’absentéisme ?
	\end{example}
\end{frame}

\begin{frame}
	\frametitle{Conseil}
	\begin{itemize}
		\item Formulez une question
		\item Commençant par \emph{Comment…}
		\item (Facultatif) reformulez le style
	\end{itemize}
\end{frame}

\section{Contenu}

\begin{frame}
	\frametitle{Contenu}
	\begin{itemize}
		\item Présenter les solutions envisagées
		\item Présenter les critères pertinents et comparer
		\item Donner votre recommandation (pluralité possible !)
	\end{itemize}
	\begin{block}{Valeur ajoutée}
				\begin{itemize}
					\item Aller au-delà des évidences (gagner du temps, c’est bien…)
					\item[$\Rightarrow$] Données \alert{chiffrées} !
				\end{itemize}
	\end{block}
	Si vous ne pouvez pas récolter des données ?
	\begin{itemize}
		\item Expliquer ce qu’il faudrait collecter (valeur de l’information)
		\item Estimations raisonnables
		\item Travail sur base d’hypothèses
	\end{itemize}
\end{frame}

\begin{frame}
	\frametitle{S’appuyer sur l’existant}
	\begin{itemize}
		\item Ne pas réinventer la roue !
		\item Chercher et citer les travaux existants
		\item Qu’avez-vous cherché ?
		\item Que disent-ils de pertinent (ou pourquoi n’est-ce pas pertinent) ?
	\end{itemize}
	\begin{block}{Sources}
		\begin{itemize}
			\item Ouvrages de référence : livres sur le sujet ou sujets connexes
			\item Cabinets de conseil
			\item Presse spécialisée
			\item Normes et standards
			\item Publications scientifiques
		\end{itemize}
	\end{block}
\end{frame}

\begin{frame}
	\frametitle{Citer}
	\begin{itemize}
		\item Citer dans le texte !
		\item Utiliser des pointeurs vers la bibliographie (habituellement : noms des auteurs et date)
		\item Ne pas laisser la possibilité de croire que vous essayez de vous approprier le travail des autres
	\end{itemize}
	\begin{example}[Pointeur]
		Comme l’indiquent Machin et Bidule (2018), …
	\end{example}
\end{frame}

\begin{frame}
	\frametitle{Suggestion de plan}
	\begin{enumerate}
		\item Introduction
		\item Contexte
		\item État de l’art
		\item Solutions proposées
		\item Analyse
		\item Préconisations
		\item Références générales
		\item Bibliographie
		\item Glossaire (facultatif)
	\end{enumerate}
\end{frame}

\section{En pratique}
\begin{frame}
	\frametitle{Calendrier}
	\begin{itemize}
		\item Fin janvier : choix de la problématique
		\item Fin mars : Plan détaillé
		\item Mi mai : État de l’art
		\item Fin aout : Remise du mémoire
		\item Début septembre : soutenances
	\end{itemize}
\end{frame}

\begin{frame}
	\frametitle{Soutenance}
  \begin{columns}
    \column{\dimexpr\textwidth+1mm}
		\begin{itemize}
			\item À Dauphine
			\item Jury : tutrice enseignante, représentante CFA, présidente du jury
			\item Présence appréciée de la personne encadrant en entreprise
			\item Déroulement : 20 minutes de présentation et 10 minutes de questions du jury
		\end{itemize}
		\end{columns}
\end{frame}

\begin{frame}[allowframebreaks]
	\frametitle{Exemples de sujets}
	\begin{itemize}
		\item Mise en place d’une architecture Big Data pour l’analyse des données Twitter en temps réel
   \item Comment améliorer l’interopérabilité et l’échange des données entre les applications du SI ?
   \item L’informatique Décisionnelle temps réel pour l’amélioration de la qualité de services
   \item Quels sont les facteurs de viabilité d’une mise en place d’une solution de test automatisée dans un projet SI ?
   \item Comment la pertinence sémantique dans les moteurs de recherche peut se rendre utile en entreprise ?  
   \item Le NoSQL peut-il remplacer les SGBD-R pour effectuer des calculs statistiques sur des données semi-structurées que sont les logs ?
   \item Comment améliorer l’efficacité des démarches de gestion de projets ? 
   \item L’informatique décisionnelle favorise-t-elle le pilotage de la gestion d’actifs ?
   \item Modélisation de données facilitant l’analyse et l’accès aux informations dans un but prédictif
   \item Comment fiabiliser les impacts sur les applications en interaction lors d’une évolution de solution ? 
   \item Machine learning et Data Mining : pour un modèle efficace de détection de fraude
   \item Comment évaluer le désengagement des salaries d’une entreprise ? 
	\end{itemize}
\end{frame} 

\begin{frame}[plain]
	\addtocounter{framenumber}{-1}
	\begin{center}
		\huge
		\textit{Bon travail !}
	\end{center}
\end{frame}

\end{document}

\appendix
\AtBeginSection{
}

\begin{frame}[allowframebreaks]
	\frametitle{\refname}
 	\bibliography{zotero}
\end{frame}

\clearpage\pdfbookmark{License}{License}
\begin{frame}[plain]
	\frametitle{License}
	This presentation, and the associated \LaTeX{} code, are published under the \href{https://opensource.org/licenses/MIT}{MIT license}. Feel free to reuse (parts of) the presentation, under condition that you cite the author.
	
	Credits are to be given to \hrefblue{https://www.lamsade.dauphine.fr/~ocailloux/}{Olivier Cailloux}, Université Paris-Dauphine.
\end{frame}
\addtocounter{framenumber}{-1}
\end{document}

\begin{frame}
	\frametitle{Title}
	\begin{block}{Block}
%		\setlength\abovedisplayskip{1 ex}% reduce space above equations
		\begin{itemize}
			\item Item
		\end{itemize}
	\end{block}
	\begin{itemize}
		\item Item
	\end{itemize}
\end{frame}

\begin{frame}
	\frametitle{Title}
	\begin{itemize}
		\item Item
	\end{itemize}
\end{frame}

